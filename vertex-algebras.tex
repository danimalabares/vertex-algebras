\input{preamble}

\begin{document}

\title{Vertex Algebras}
\maketitle

\phantomsection
\label{section-phantom}
\hfill
\href{http://github.com/danimalabares/vertex-algebras}
{github.com/danimalabares/vertex-algebras}

\tableofcontents

\section{Cartan subalgebra, Cartan matrix and Serre relations}
\label{section-Cartan-subalgebra}

Kac-Moody algebras are Lie algebras, whose definition is motivated by the
structure of finite-dimensional simple Lie algebras over $\mathbb{C}$.

Let $\mathfrak{g}$ be a finite-dimensional semisimple Lie algebra over
$\mathbb{C}$. Then $\mathfrak{g}$ has a {\it Cartan subalgebra} 
$\mathfrak{h}\subset \mathfrak{g}$ (abelian + …) 
(see \cite[Definition 8.2]{KLAL}). 
Fixing $\mathfrak{h}\subset\mathfrak{g}$ gives a 
{\it root space decomposition} (see \cite[Proposition 8.5]{KLAL})
$$
\mathfrak{g}=\mathfrak{h}\oplus \bigoplus_{\alpha \in \Delta}\mathfrak{g}_\alpha
$$
where $\Delta \subset \mathfrak{h}^*$ linear dual, and, by definition
$$
\mathfrak{g}_\alpha=\{X \in \mathfrak{g}|
[H,X]=\alpha(H)X\; \forall H \in \mathfrak{h}\}
$$
Turns out the $\mathfrak{g}_\alpha$ are all 1-dimensional, 
though this property is lost when we go to Kac-Moody algebras.
$$
[\mathfrak{g}_\alpha,\mathfrak{g}_\beta] \subset \mathfrak{g}_{\alpha+\beta}
$$
The Killing form  $\kappa:\mathfrak{g} \times \mathfrak{g} \to \mathbb{C}$,
 $\kappa(x,y)=\text{Tr}_\mathfrak{g}\text{ad}(x)\text{ad}(y)$ is nondegenerate. 
``This is kind of the definition of semisimple.'' 
(Think of $\mathfrak{h}$ as $\mathfrak{g}_0$, btw.)

$\kappa |_{\mathfrak{g}_\alpha \times \mathfrak{g}_\beta}\neq 0$ only when 
$\beta=-\alpha$. $\kappa |_{\mathfrak{h}\times \mathfrak{h}}$ is non-degenerate. 
 This gives a linear isomorphism 
$\mathfrak{h} \xrightarrow{\nu} \mathfrak{h}$ via 
$\nu(H)(H') = \kappa(H,H')$.

\medskip\noindent
So, $\mathfrak{h}^*$ comes with a non-degenerate bilinear form.

The {\it reflection} $r_\alpha:\mathfrak{h}\to \mathfrak{h}^*$ in 
$\alpha \in \mathfrak{h}^*$ (usually a root) is 
$r_\alpha(\lambda)=\lambda- 2\frac{(\lambda,\alpha)}{(\alpha,\alpha)}
\cdot \alpha.$

``Classify root systems […] classify semisimple Lie algebras'' 
It is a fact that $r_\alpha (\Delta)=\Delta$ for all $\alpha \in \Delta$, 
which motivates the definition of {\it root system} 
(see \cite[Definition 15.1]{KLAL}) 
and permits classification. (See \cite[Lecture 17]{KLAL} for comments on
correspondence of root systems and semisimple Lie algebras.)

\begin{example}
\label{example-root-system-sl2}
$\mathfrak{g}=\mathfrak{sl}_2$, $\mathfrak{h}=$ diagonal matrices
$$
\begin{pmatrix}
1&&\\ 
&-1&\\
& & 0
\end{pmatrix}
\qquad 
\begin{pmatrix}
0&&\\ 
&1&\\
& & -1
\end{pmatrix}
$$
is a basis of $\mathfrak{h}$. There are 6 roots vectors
$$
E_{12}=
\begin{pmatrix}
0 & 1 & 0\\
0 & 0 & 0\\
0 & 0 & 0
\end{pmatrix}
$$
$E_{23},E_{13}$, etc.
\end{example}

\begin{exercise}
\label{exercise-basis-of-h}
$[H_1,E_{12}]=2E_{12}$, $[H_2,E_{12}]=-E_{12}$, $\alpha_{12}=(2,-1)$.
\end{exercise}

[Drawing of roots]

Notions of {\it positive roots} and {\it simple roots} 
(set of $\text{rank}\mathfrak{g}$ simple roots has $\ell$ elements,
 where $\ell=\dim(\mathfrak{h}^*)$.  (See \cite[Definition 17.1]{KLAL}.)
This will also fail for Kac-Moody algebras more generally). 
Next write the {\it Cartan matrix} (see \cite[Definition 17.2]{KLAL})
$$
A=(a_{ij}),\qquad  a_{ij}=2\frac{(\alpha_i,\alpha_j)}{(\alpha_i,\alpha_i)}
$$
for $1\leq i,j \leq \ell$.

\begin{example}
\label{example-Cartan-matrix-sl3}
$\mathfrak{sl}_3$. [Picture, hexagonal pattern]. 
$(\alpha_1,\alpha_1)=(\alpha_2,\alpha_2)=2,(\alpha_1,\alpha_2)=-1$, so
$$
A=\begin{pmatrix}
2&-1\\ 
-1&2
\end{pmatrix}
$$
\end{example}

\begin{example}
\label{example-Cartan-matrix-sl5}
$\mathfrak{sl}_5$. [Picture, square pattern]. $|\alpha_2|=1$, $|\alpha_1|=2$,
$(\alpha_1,\alpha_2)=-2$, so
$$
A=\begin{pmatrix}
2&-1\\ 
-2&2
\end{pmatrix}
$$
\end{example}

\begin{remark}
\label{remark-}
Since $\mathfrak{g}_\alpha$ is 1-dimensional, set
$\mathfrak{g}_\alpha=\mathbb{C}E_\alpha$ and 
$E_i=E_{\alpha_i}$, $i=1,2,\ldots,\ell$ (simple root vectors). 
It turns out that 
$$
\text{ad}(E_i)^{1-a_{ij}}E_j=0.
$$
This is called a {\it Serre relation}.
\end{remark}

\section{Some infinite dimensional Lie algebras}
\label{section-some-infinite-dimensional-Lie-algebras}
Let $\mathfrak{g}$ be a finite-dimensional semisimple Lie algebra, and define
the {\it loop algebra} 
\begin{align*}
L\mathfrak{g}&=\mathfrak{g}[t,t^{-1}],\text{ (with basis 
$at^m|\substack{\text{$a \in $ a basis of $\mathfrak{g}$} \\ m \in \mathbb{Z}}$
)}\\
&=\mathfrak{g} \otimes _{\mathbb{C}}\mathbb{C}[t,t^{-1}]
\end{align*}
with the Lie bracket
$$
[at^m,bt^n]=[a,b]t^{m+n}.
$$
``This construction is absurdely general --- we don't need $\mathfrak{g}$ to be
semisimple […]''

\medskip\noindent
Take $\mathfrak{g}=\mathfrak{sl}_2$. Recall that
$$
E=\begin{pmatrix}
0&1\\ 
0&0
\end{pmatrix},\quad 
H=\begin{pmatrix}
1&0\\ 
0&-1
\end{pmatrix},\quad 
F=\begin{pmatrix}
0&0\\ 
1&0
\end{pmatrix}
$$
[Picture with $F,H,E,Ft,Ht,Et,Et^2$…]
$E$ was a root vector, corresponding to the unique root in $\mathfrak{sl}_2$,
call it $\alpha_1$. We seem to have a second simple root $\alpha_0$,
 corresponding to $Ft$.

This looks like it wants to have a Cartan matrix
 $$
A=\begin{pmatrix}
2&-2\\ 
-2&2
\end{pmatrix}
$$
We will indeed recover (a variant of) $L\mathfrak{g}$ as a Lie algebra 
``built from'' $A=\begin{pmatrix}
2&-2\\ 
-2&2
\end{pmatrix}$,
 a Kac-Moody algebra. 
But note first, $\mathfrak{h}=\mathbb{C}H$ is too small.
``Problem with $\alpha_0$ and $\alpha_1$ being linearly independent  …''

\begin{exercise}
\label{exercise-Lie-algebra-d}
Consider $L\mathfrak{g}\oplus\mathbb{C}d$, and set 
$[d,at^m]=mat^m$, $[d,d]=0$. Check this defines a Lie algebra.
\end{exercise}

\begin{proof}
Skew-commutativity, i.e. for all $x \in L\mathfrak{g}\oplus\mathbb{C}d$,
\begin{equation}
\label{equation-skew-commutativity}
[x,x]=0,
\end{equation}
is immediate from 
skew commutativity in $L\mathfrak{g}$ and 
the hypothesis that $[d,d]=0$.

To confirm Jacobi identity, i.e. that for all 
$x,y,z \in L\mathfrak{g}\oplus\mathbb{C}d$
\begin{equation}
\label{equation-Jacobi-algebra-d-identity}
[x,[y,z]]+[y,[z,x]]+[z,[x,y]]=0,
\end{equation}
notice that since this is a cyclic sum on $x,y,z$ 
we only need to consider three elements in
$L\mathfrak{g} \oplus \mathbb{C}d$ up to cyclic permutation.
The cases in which the three elements are either in $L\mathfrak{g}$ or in
$\mathbb{C}d$ are obvious, so that there
are only two interesting possibilities:
\begin{align}
x=d,\qquad y&=at^m,\qquad z=bt^n
\label{equation-Jacobi-algebra-d-case-1}\\
x=d,\qquad y&=d,\qquad z=at^n
\label{equation-Jacobi-algebra-d-case-2}
\end{align}
Case \ref{equation-Jacobi-algebra-d-case-1} gives
\begin{align*}
& [d,[at^m,bt^n]]+[at^m,[bt^n,d]]+[bt^n,[d,at^m]]\\
&=[d,[a,b]t^{m+n}]+[at^m,-nbt^n]+[bt^n,mat^m]\\
&=(m+n)[a,b]t^{m+n}-n[a,b]t^{m+n}+m[b,a]t^{m+n}\\
&=(m+n)[a,b]t^{m+n}-n[a,b]t^{m+n}-m[a,b]t^{m+n}\\
&=(m+n)[a,b]t^{m+n}-(m+n)[a,b]t^{m+n}=0.
\end{align*}
Case \ref{equation-Jacobi-algebra-d-case-2} gives
\begin{align*}
&[d,[d,at^m]]+[d,[at^m,d]]+[at^m,[d,d]]\\
&=[d,mat^m]+[d,-mat^m]=0.
\end{align*}

\end{proof}

The Kac-Moody algebra turns out to be, not quite this, but slightly larger
still.

Recall that an {\it invariant bilinear form} $(\cdot,\cdot)$ on a Lie algebra
$\mathfrak{g}$ is a bilinear form such that
\begin{equation}
\label{equation-invariant-bilinear-form}
([a,b],c)=(a,[b,c])\qquad \forall a,b,c \in \mathfrak{g}.
\end{equation}

\begin{exercise}
\label{exercise-invariant-bilinear-form-on-simple-Lie-algebra-is-symmetric}
Prove that an invariant bilinear form on a simple Lie algebra must in fact be
symmetric.
\end{exercise}

\begin{proof}
It's enough to show that $\mathfrak{g}$ is
{\it perfect}, i.e. that $[\mathfrak{g},\mathfrak{g}]=\mathfrak{g}$. In this
case, let $a,b \in \mathfrak{g}$ and suppose that $b=[x,y]$. Then
\begin{align*}
(a,b)&=(a,[x,y])=(a,-[y,x])=(-[a,y],x)=([y,a],x)\\
&=(y,[a,x])=(y,-[x,a])=(-[y,x],a)=([x,y],a)=(b,a)
\end{align*}
To confirm that $\mathfrak{g}$ is perfect just observe that
$[\mathfrak{g},\mathfrak{g}]$ is a nontrivial ideal of $\mathfrak{g}$.
\end{proof}

\begin{definition}
\label{definition-affine-Lie-algebra}
Given $\mathfrak{g}$ simple, with 
$(\cdot,\cdot):\mathfrak{g} \times \mathfrak{g} \to \mathbb{C}$ 
invariant bilinear form, the {\it affine Lie algebra} is 
$$
\hat{\mathfrak{g}}=L\mathfrak{g} \oplus \mathbb{C}K,
$$
with $[K,\hat{\mathfrak{g}}]=0$, and 
$[at^m,bt^n]=[a,b]t^{m+n}+m(a,b)\delta_{m,-n}K$.
\end{definition}

``For the construction to work it doesn't actually have to be nondegenerate.''

\begin{exercise}
\label{exercise-affine-Lie-algebra-is-Lie-algebra}
Check that the affine Lie algebra $\hat{\mathfrak{g}}$ is a Lie algebra.
\end{exercise}

\begin{proof}
(Skew-commutativity.) 
Since $[K,\hat{\mathfrak{g}}]=0$ and $K \in \hat{\mathfrak{g}}$, 
it is immediate that $[K,K]=0$. 
For the case of an element in $L\mathfrak{g}$, we see that
$[at^m,at^m]=0$ by skew-commutativity of the bracket in $\mathfrak{g}$ 
and the Kronecker delta.
 
(Jacobi identity.) As in Exercise \ref{exercise-Lie-algebra-d}, 
any choice of $x,y,z$ involving $K$ is immediate by $[K,\hat{\mathfrak{g}}]=0$.
Thus the only interesting case is
for Jacobi identity consider the cases
\begin{align*}
& [at^m,[bt^n,ct^\ell]]+[bt^n,[ct^\ell,at^m]]+[ct^\ell,[at^m,bt^n]]\\
&=[at^m,[b,c]t^{n+\ell}+n(b,c)\delta_{n,-\ell}K]\\
&+[bt^n,[c,a]t^{\ell+m}+\ell(c,a)\delta_{\ell,-m}K]\\
&+[ct^\ell,[a,b]t^{m+n}+m(a,b)\delta_{m,-n}K]\\
&=[at^m,[b,c]t^{n+\ell}]+[at^m,n(b,c)\delta_{n,-\ell}K]\\
&+[bt^n,[c,a]t^{\ell+m}]+[bt^n,\ell(c,a)\delta_{\ell,-m}K]\\
&+[ct^\ell,[a,b]t^{m+n}]+[ct^\ell,m(a,b)\delta_{m,-n}K]\\
&=[a,[b,c]]t^{m+(n+\ell)}+m(a,[b,c])\delta_{m,-(n+\ell)}K\\
&+[b,[c,a]]t^{n+(\ell+m)}+n(b,[c,a])\delta_{n,-(\ell+m)}K\\
&+[c,[a,b]]t^{\ell+(m+n)}+\ell(c,[a,b])\delta_{\ell,-(m+n)}K=0
\end{align*}
It is clear that we obtain a Jacobi equation on $\mathfrak{g}$. To see that the
remaining terms vanish, notice that the condition on
the Kronecker delta in its three appearances is the same,
namely, $m+n+\ell=0$. In this case, we only need to check
that $(a,[b,c])=(b,[c,a])=(c,[a,b])$ to conclude.
This follows from the invariance of $(\cdot,\cdot)$
and the fact that $\mathfrak{g}$ simple using Exercise 
\ref{exercise-invariant-bilinear-form-on-simple-Lie-algebra-is-symmetric}.
\end{proof}

We also have
\begin{definition}
\label{definition-extended-affine-Lie-algebra}
The {\it extended affine Lie algebra} is 
$$
\tilde{\mathfrak{g}}=L\mathfrak{g} \oplus \mathbb{C}K \oplus \mathbb{C}d,
$$ 
with $[d,at^m]=m a t^m$ as before, and 
$[K,d]=0$.
\end{definition}

The extended affine Lie algebra is an example of a Kac-Moody algebra.

\begin{exercise}[For those who like geometry]
\label{exercise-for-those-who-like-geometry}
Let $R=\mathbb{C}[t,t^{-1}]$. 
If $D \in \text{Der}(R)$, then
$L\mathfrak{g} \oplus \mathbb{C}d$ is a Lie algebra with 
$[d,a \otimes r]=a\otimes D(r)$. 
Is $(\mathfrak{g}\otimes R)\oplus \text{Der}(R)$ a Lie algebra? (The Lie alegra
$L\mathfrak{g} \oplus \mathbb{C}d$ from Exercise \ref{exercise-Lie-algebra-d} is
a particular case, for $D=t\frac{d}{dt}$.)
\end{exercise}

\begin{proof}
Checking that $L\mathfrak{g} \oplus \mathbb{C}d$ is a Lie algebra with 
$[d,a\otimes r]=a \otimes D(r)$ is similar to Exercise
\ref{exercise-Lie-algebra-d}: skew-commutativity is immediate from
skew-commutativity in each of the components, while 
Jacobi identity is verified in two cases. For $x=y=d$ and $z=a \otimes r$ 
we quickly obtain
\begin{align*}
&[x,[y,z]]+[y,[z,x]]+[z,[x,y]]\\
&=[d,[d,a \otimes r]]+[d,[a \otimes r,d]]+[a \otimes r,[d,d]]\\
&=[d,a \otimes D(r)]+[d,-a \otimes D(r)]=0.
\end{align*}

And for $x=d$, $y=a \otimes r$ and $z=b \otimes s$, we get
\begin{equation}
\label{equation-Jacobi-derivation}
\begin{aligned}
&[x,[y,z]]+[y,[z,x]]+[z,[x,y]]\\
&=[d,[a\otimes r,b \otimes s]]+[a\otimes r,[b \otimes s,d]]
+[b \otimes s,[d,a\otimes r]]\\
&=[d,[a,b]\otimes rs]+[a\otimes r,-b\otimes D(s)]
+[b\otimes s,a\otimes D(r)]\\
&=[a,b]\otimes D(rs)-[a,b]\otimes rD(s)+[b,a]\otimes sD(r)=0.
\end{aligned}
\end{equation}

\medskip\noindent
To check whether $(\mathfrak{g}\otimes R)\oplus\text{Der}(R)$ is a Lie algebra 
first put the Lie bracket on $\text{Der}(R)$ as
$[D,D_1]=DD_1-D_1D$. It is clear that this bracket is skew-commutative. 
Jacobi identity reads
\begin{align*}
&[D,[D_1,D_2]]+[D_1,[D_2,D]]+[D_2,[D,D_1]]\\
&=[D,D_1D_2-D_2D_1]+[D_1,D_2D-DD_2]+[D_2,DD_1-D_1D]\\
&=D(D_1D_2-D_2D_1)-(D_1D_2-D_2D_1)D+D_1(D_2D-DD_2)\\
&-(D_2D-DD_2)D_1+D_2(DD_1-D_1D)-(DD_1-D_1D)D_2\\
&=DD_1D_2-DD_2D_1-D_1D_2D+D_2D_1D+D_1D_2D-D_1DD_2\\
&-D_2DD_1+DD_2D_1+D_2DD_1-D_2D_1D-DD_1D_2+D_1DD_2=0.
\end{align*}
Now put the bracket on $(\mathfrak{g} \otimes  R)\oplus\text{Der}(R)$ as
$[D,a \otimes  r]=a \otimes D(r)$. Skew-commutativity is immediate. 
Jacobi identity for $x=D,y=a\otimes r$ and $z=b\otimes s$ is identical to the
computation \ref{equation-Jacobi-derivation}. In the case
$x=D$, $y=D_1$ and $z=a\otimes r$, we get
\begin{align*}
&[D,[D_1,a\otimes r]]+[D_1,[a\otimes r,D]]+[a\otimes r,[D,D_1]]\\
&=[D,a\otimes D_1(r)]+[D_1,-a\otimes D(r)]+[a\otimes r,[D,D_1]]\\
&=a\otimes DD_1(r)-a\otimes D_1D(r)-a\otimes [D,D_1](r)=0
\end{align*}
\end{proof}

\section{Kac-Moody algebras}
\label{section-Kac-Moody-algebras}

Recall the notion of the free Lie algebra on a 
vector space $V$ of generators 
(or a set $X$, think of $V$ as a vector space with basis $X$):

\begin{definition}
\label{definition-free-Lie-algebra}
The {\it free Lie algebra} on $V$ is characterized by the universal property
$$
\xymatrix{
V\ar[rr]^f\ar[dr]_i&&\mathfrak{g}\\
& F(V)\ar[ur]_{\exists !\tilde{f}}
}
$$
That is, for any linear map $f:V \to \mathfrak{g}$ with $\mathfrak{g}$ Lie
algebra, there exists a unique $\tilde{f}$ homomorphism of Lie algebras 
$F(V)\to \mathfrak{g}$ such that $\tilde{f} \circ i=f$.
$$
\Hom_{\text{Lie}}(F(V),\mathfrak{g})=\Hom_{\text{Vec}}(V,\mathfrak{g})
$$
naturally.
\end{definition}

That is, $F$ and the 
forgetful functor $G:\underline{\text{Lie}}\to \underline{\text{Vec}}$
are adjoint:
$$
\Hom_{\underline{\text{Lie}}}(F(V),\mathfrak{g})
\xrightarrow{\simeq }
\Hom_{\underline{\text{Vec}}}(V,G(\mathfrak{g}))
$$

\medskip\noindent
{\bf A realisation of $F(V)$.} Let 
$$
T(V)=\mathbb{C} \oplus V \oplus V^{\otimes 2}\oplus V^{\otimes 3}\oplus\ldots
$$
be the tensor algebra of $V$. 

Then inside $T(V)$ consider $F(V)$ the span of iterated commutators of elements
of  $V$.

\begin{proposition}
\label{proposition-this-realises-the-free-Lie-algebra}
This realises the free Lie algebra.
\end{proposition}

\begin{proof}
In online notes.
\end{proof}

\medskip\noindent
In the finite dimensional simple case, we had
$$
a_{ij}=\frac{2(\alpha_i,\alpha_j)}{(\alpha_i,\alpha_i)},
$$
which we think also as $\alpha_i,\alpha_j \in \mathfrak{h}^*$, 
and $\alpha_i^\vee=\frac{2}{(\alpha_i,\alpha_i)}\nu^{-1}(\alpha_i) 
\in \mathfrak{h}.$

Clearly, $\alpha_{ii}=2$ for all $i$. 
$a_{ij}$ might not equal $a_{ji}$, but certainly $a_{ij}=0 \iff a_{ji}=0$. 
And $\forall  i \neq j$, $a_{ij} \leq 0$.

\medskip\noindent
{\bf One further property.} Set
$$
\varepsilon_i=\frac{2}{(\alpha_i,\alpha_i)},\quad \text{and}\quad 
D=\substack{\text{diagonal matrix} \\ \text{with entries }\varepsilon_i}
$$
Then $A=DB$, where $B=((\alpha_i,\alpha_j))$ is symmetric. 
If a matrix $A$ is equal to $(\text{diag})(\text{symm})$, we call it
 {\it symmetrizable}.

\begin{definition}
\label{definition-generalized-Cartan-matrix}
A {\it generalized Cartan matrix} is an integer matrix $A=(a_{ij})$ 
which is
\begin{itemize}
\item symmetrizable,
\item $a_{ii}=2$ for all $i$,
\item $a_{ij}=0 \iff a_{ji}=0$,
\item $a_{ij}\leq 0$ for $i \neq j$.
\end{itemize}
\end{definition}

\begin{definition}
\label{definition-realisation}
A {\it realisation} of a generalized Cartan matrix is a complex vector space
 $\mathfrak{h}$, and two sets
\begin{align*}
\Pi^\vee&=\{\alpha_1^\vee, \alpha_2^\vee,\ldots,\alpha_n^\vee\},\quad
\text{and},\\
\Pi&=\{\alpha_1,\alpha_2,\ldots,\alpha_n\}
\end{align*}
such that $\left<\alpha_i^\vee,\alpha_j\right>=a_{ij}$, $1\leq i,j\leq n$.
\end{definition}

\begin{exercise}
\label{exercise-realisation}
$\dim(\mathfrak{h})\geq 2n-\text{rank}(A)$.
\end{exercise}

\begin{proof}
For $A=\begin{pmatrix}
2&-2\\ 
-2&2
\end{pmatrix}$, a realisation is given by
$$
\Pi^\vee=\{H_1,H_0\},\qquad \Pi=\{\alpha_0,\alpha_1\}
$$
\begin{align*}
\mathfrak{h}&=\mathbb{C}H,\mathbb{C}d,\mathbb{C}K,\\
\mathfrak{h}^*&=\mathbb{C}\alpha_1+\mathbb{C}\delta+\mathbb{C}\Lambda_0
\end{align*}
(Canonical dual, $\left<\alpha_1,H\right>=2$, 
$\left<\delta,d\right>=1=\left<\Lambda_0,K\right>$, every other pairing $0$.)

Then
$$
\begin{cases}
\alpha_1=\alpha_1\\
\alpha_0=\delta-\alpha_1
\end{cases}\qquad 
\begin{cases}
\alpha_1^\vee=H \\
\alpha_0^\vee = K-H
\end{cases}
$$
So we obtain
\begin{align*}
\left<\alpha_0^\vee,\alpha_1\right>&=\left<K-H,\alpha_1\right>=2\\
\left<\alpha_1^\vee,\alpha_0\right>&=\left<H,\delta-\alpha_1\right>=-2\\
\left<\alpha_0^\vee,\alpha_0\right>&=\left<K-H,\delta-\alpha_1\right>=+2
\end{align*}

\end{proof}

\medskip\noindent
Finally let's define Kac-Moody algebras.

Let $A$ be a generalized Cartan matrix. Let 
$$
\tilde{\mathfrak{n}}_+=F(e_1,\ldots,e_n),
$$ 
the free Lie algebra on $n$ generators, and similarly
$$
\tilde{\mathfrak{n}}_-=F(f_1,\ldots,f_n).
$$
Let $\mathfrak{h}$ be a realisation of $A$. 
Set $\tilde{\mathfrak{g}}(A)
=\tilde{\mathfrak{n}}_- \oplus \mathfrak{h} \oplus \tilde{\mathfrak{n}}_+$.

Make $\tilde{\mathfrak{g}}(A)$ a Lie algebra by defining
\begin{itemize}
\item $[\mathfrak{h},\mathfrak{h}]=0$,
\item $\forall  H \in \mathfrak{h}$, $[H,e_i]=\left<\alpha_i,H\right>e_i
=\alpha_i(H)e_i$. And similarly, $[H,f_i]=-\alpha_i(H)f_i$.
\item $[e_i,f_j]=\delta_{ij}\alpha_i^\vee$.
\end{itemize}
Then $\tilde{\mathfrak{g}}(A)$ is a Lie algebra (though not yet the Kac-Moody
algebra). See Kac, \cite[Thorem 1.2]{IDLA}.

\begin{remark}
\label{remark-lattice}
In $\mathfrak{h}$ we have a lattice
\begin{align*}
Q^\vee&=\mathbb{Z}\alpha_1^\vee+\ldots+\mathbb{Z}\alpha_n^\vee,\quad
\text{and}\\
Q&=\mathbb{Z}\alpha_1+\ldots+\mathbb{Z}\alpha_n\text{ in }\mathfrak{h}^*
\end{align*}
(root and coroot lattices). $\tilde{\mathfrak{g}}(A)$ is naturally $Q$-graded,
with 
$$
\tilde{\mathfrak{g}}(A)_\beta=\text{span}\{
\text{commutators of $e_i$ with $\sum \alpha_i=\beta$}\}.
$$
$\tilde{g}(A)=\mathfrak{h}$.

\begin{theorem}[Gabber-Kac]
\label{theorem-Gabber-Kac}
Denote by $I \subset \tilde{\mathfrak{g}}(A)$ the maximal $Q$-graded ideal, such
that $I \cap\mathfrak{h}=\{0\}$. 
Then $I$ is generated by the Serre relations 
$$
\text{ad}(e_i)^{1-a_{ij}}e_j\qquad \text{and}\qquad 
\text{ad}(f_i)^{1-a_{ij}}f_j,\; i\neq j.
$$
\end{theorem}

\begin{proof}
\cite[Theorem 9.11]{IDLA}.
\end{proof}

(The existence of the ideal $I$ does not need the theorem; 
the importance of the theorem is providing an
expression for the generators.)

\begin{definition}
\label{definition-Kac-Moody-algebra}
The {\it Kac-Moody algebra} $\mathfrak{g}(A)$ is $\tilde{\mathfrak{g}}(A)/I$.
\end{definition}

\end{remark}

\section{Affine Kac-Moody algebras}
\label{section-affine-Kac-Moody-algebras}

Let $\mathfrak{g}$ be a finite-dimensional simple Lie algebra, 
with $(\cdot,\cdot):\mathfrak{g} \times \mathfrak{g} \to \mathbb{C}$ 
invariant bilinear form,
$$
([x,y],z)=(z,[y,z])\qquad \forall x,y,z \in \mathfrak{g}
$$
(Eg. the Killing form  
$\kappa(x,y)=\text{Tr}_\mathfrak{g}\text{ad}(x)\text{ad}(y)$ is invariant.)

Typically we normalise $(\cdot,\cdot)$ so that $(\alpha,\alpha)=2$ for the long
roots of $\mathfrak{g}$.

Then $\hat{\mathfrak{g}}=\mathfrak{g}[t,t^{-1}]\oplus\mathbb{C}K$ 
(affine Lie algebra),
$$
[at^m,bt^n]=[a,b]t^{m+n}+m\delta_{m,-n}(a,b)K,\qquad [K,\hat{\mathfrak{g}}]=0
$$
and $\tilde{\mathfrak{g}}=\hat{\mathfrak{g}}\oplus\mathbb{C}d$, 
$[d,K]=0$,  $[d,at^m]=mat^m$, 
(affine Kac-Moody algebra or ``extended affine Lie algebra'')

\begin{theorem}
\label{theorem-gtilde-is-a-Kac-Moody-algebra}
$\tilde{\mathfrak{g}}$ is a Kac-Moody algebra.
\end{theorem}

Let $\mathfrak{g}=\mathfrak{h}\oplus \bigoplus_{\alpha \in
\Delta}\mathfrak{g}_\alpha$, ($\mathfrak{g}_\alpha=\mathbb{C}E_\alpha$.)

{\bf The simple roots and coroots.}
$\tilde{\mathfrak{h}}=\mathfrak{h}\oplus\mathbb{C}K\oplus\mathbb{C}d$. 
We identify $\tilde{\mathfrak{h}}^*$ with 
 $\mathfrak{h}^*\oplus\mathbb{C}\Lambda_0\oplus\mathbb{C}\delta$ where
\begin{align*}
\Lambda_0(\mathfrak{h})&=\delta(\mathfrak{h})=0\\
\Lambda_0(d)&=\delta(K)=0\\
\Lambda_0(K)&=\delta(d)=1
\end{align*}
The {\it real coroots} are
$$
\hat{\Delta}^{V,re}=\{E_\alpha t^m|\alpha \in \Delta, m \in \mathbb{Z}\}
$$
and there are also imaginary roots and coroots
$$
\hat{\Delta}^{V,im}=\{Ht^m|H \in \mathfrak{h}, m \in \mathbb{Z}\setminus\{0\}\}
$$
Roots:
\begin{align*}
\hat{\Delta}^{re}&=\{\alpha+m\delta|\alpha \in \Delta, m \in \mathbb{Z}\}\\
\hat{\Delta}^{im}&=\{m\delta|m \neq 0\}
\end{align*}
$Xt^m$:
\begin{align*}
[H,Xt^m]&=[H,x]t^m,\qquad H \in \mathfrak{h}\\
[K,xt^m]&=0\\
[d,xt^m]&=m x t^m
\end{align*}
so it $x \in \mathfrak{g}_\alpha$, $x t^m \in 
\tilde{\mathfrak{g}}_{\alpha+m\delta}$.

\medskip\noindent
The invariant bilinear form $(\cdot,\cdot)$ from $\mathfrak{g} \times
\mathfrak{g}$ extends uniquely to
$(\cdot,\cdot):\tilde{\mathfrak{g}}\times\tilde{\mathfrak{g}}\to \mathbb{C}$.

$(d,d)=(K,K)=0$, $(d,K)=1$ and $(d,\mathfrak{h})=(K,\mathfrak{h})=0$.

\medskip\noindent
So, in $\tilde{\mathfrak{h}}^*$:
\begin{align*}
(\Lambda_0,\Lambda_0)&=(\delta,\delta)=0\\
(\Lambda_0,\mathfrak{h}^*)&=(\delta,\mathfrak{h}^*)=0\\
(\Lambda_0,\delta)&=1.
\end{align*}
Hence, $|\alpha+m \delta|^2=|\alpha|^2$, $|m \delta|^2=0$.

\begin{example}
\label{example-sl2hat}
$\widetilde{\mathfrak{sl}_2}, \tilde{\mathfrak{h}}^*
=\text{span}\{\alpha,\Lambda_0,\delta\}$ with Gram matrix …
\end{example}

We can make a choice of positive roots,
$$
\hat{\Delta}_+=\{\alpha+m \delta |\alpha \in \Delta, m>0\}
\cup \{m \delta|m>0\} \cup \Delta_+
$$
Obviously, if $\alpha \in \Delta_+$ is simple, 
$\alpha \in \hat{\Delta}_+$ is simple.

\medskip\noindent
{\bf Notation.} Let $\theta \in \Delta_+$ be a the highest root. ($\not \exists
\alpha \in \Delta_+$ such that $\alpha-\theta \in \mathbb{Z}_+ \Delta_+$.) and
 $\alpha=\delta-\theta$.

Then $\alpha_0 \in \hat{\Delta}_+$ is simple and the set of simple roots is
$\hat{\Pi}=\{\alpha_0,\underbrace{\alpha_1,\ldots,\alpha_\ell}
_{\substack{\text{the finite} \\ \text{simple roots}}} \}.$
where $\ell=\text{rank}(\mathfrak{g})$.

The coroot corresponding to $\alpha_0$ is 
$$
\alpha^\vee_0=K-\theta^\vee, \qquad
\theta^\vee=\frac{2}{(\theta,\theta)}\nu^{-1}(\theta) \in \mathfrak{h} 
$$
$$
\text{ and }\qquad E_{\alpha_0}=E_{-\theta}t.
$$

\medskip\noindent
Now, in any Kac-Moody algebra, we have
\begin{align*}
\text{roots }\qquad \Pi&=\{\alpha_1,\ldots,\alpha_\ell\}\subset\mathfrak{h}^* \\
\text{coroots}\qquad  \Pi^\vee&=\{\alpha_1^\vee,\ldots,\alpha_\ell^\vee\}\subset
\mathfrak{h},
\end{align*}
and {\it reflections} $r_i \in \text{GL}(\mathfrak{h}^*)$,
defined by 
$$
r_i(\lambda)=\lambda-\left<\lambda,\alpha_i^\vee\right>\alpha_i.
$$

One can check that
$$
(r_i\lambda,r_i\mu)=(\lambda,\mu)\qquad \forall \lambda,\mu \in \mathfrak{h}^*
$$
The {\it Weyl group} $W$ is $\left<r_i|i=1,\ldots,\ell\right>\subset 
\text{GL}(\mathfrak{h}^*)$.

\begin{example}
\label{example-Weyl-group-sl2tilde}
For $\widetilde{\mathfrak{sl}_2}$, $r_1$ is easy,
\begin{align*}
r_1(\alpha)&=-\alpha \qquad \text{(as in $\mathfrak{sl}_2$)}\\
r_1(\delta)&=\delta,\qquad r_1(\Lambda_0)=\Lambda_0.
\end{align*}
To compute $r_0$ take an arbitrary element $m \alpha_1+k\Lambda_0+f\delta$ and
do:
\begin{align*}
r_0(m \alpha_1+k\Lambda_0+f\delta)&=m\alpha_1+k\Lambda_0+f\delta
-\left<\alpha_0^\vee,m\alpha_1+k\Lambda_0+f\delta\right>\alpha_0\\
\alpha_0=\delta-\alpha_1,\qquad & \alpha_0^\vee=K-\alpha^\vee
\end{align*}
so we obtain
$$
=m\alpha_1+k\Lambda_0+f\delta-(k-2m)(\delta-\alpha_1)
$$
$$
(k-m)\alpha_1+k\Lambda_0+(f-k+2m)\delta.
$$
Relative to basis $\{\alpha_1,\Lambda_0,\delta\}$.
$$
r_1=\begin{pmatrix}
-1&0&0\\
0&1&0\\
0&0&1
\end{pmatrix},
r_0=\begin{pmatrix}
-1&1&0\\
0&1&0\\
2&-1&1
\end{pmatrix}
\begin{pmatrix}
m\\
k\\
f
\end{pmatrix}
$$
$$
t=r_1r_0=\begin{pmatrix}
1&-1&0\\
0&1&0\\
2&-1&1
\end{pmatrix}
$$
Notice that $\delta$ is fixed by all $r_i$. Also 
$m \alpha+k \Lambda_0 +f\delta$, the {\it coefficient} of $\Lambda_0$ is fixed
by all $r_i$.

Then
\begin{align*}
t(m\alpha_1+&k\Lambda_0+f\delta)=(m-k)\alpha_1+k\Lambda_0+(f-k+2m)\delta.
\end{align*}
Think of $t$ as a translation.

The number $k$ in 
 $$
\mathfrak{h}^* \ni \hat{\lambda}=\lambda+k\Lambda_0+f\delta
$$
is called the {\it level} of $\hat{\lambda}$.

$\hat{\mathfrak{h}}=$ union of (hyper)planes of constant level which are stable
under $W$. The roots $\alpha$ are all of level $0$.

[Picture] 
``$r_1$ changes the sign of the finite path''. And $t=r_1r_0$ is a sort of
translation. Indeed, in general we can consider 
$t_{\alpha_i}=r_{\alpha_i}\circ r_0 \in W$,

$$
t_\alpha(\beta+m\delta)=\beta+(m+(\beta,\alpha_i))\delta
$$
\end{example}
One can describe the action of $t_\alpha$ on $\hat{\lambda}$ in general 
(e.g. see \cite[Chapter 6]{IDLA})

\begin{proposition}
\label{proposition-Weyl-group-of-affine-Kac-Moody-algebra}
For the affine Kac-Moody algebra $\hat{\mathfrak{g}}$
 with $\hat{W}=\left<r_0,r_1,\ldots,r_\ell\right>$ its Weyl group 
(and $W=\left<r_1,\ldots,r_\ell\right>\subset \hat{W}$ the Weyl group of
$\mathfrak{g}$), then $\hat{W} \simeq W\times t_{Q^\vee}$ (where it should be
semidirect product instead of $\times$…) where $Q^\vee$ is the
coroot lattice of $\mathfrak{g}$.
\end{proposition}

\begin{remark}
\label{remark-Weyl-groups-are-large-for-other-Kac-Moody-algebras}
For general Kac-Moody algebras, the Weyl groups are much larger, hyperbolic
reflection groups.
\end{remark}

In the affine case, $\hat{W}$ fixes level $k$, and $|\hat{\lambda}|$. One gets,
in the intersection, paraboloids [Picture of section of hyperboloid that is a
parabola].

\section{Weyl character formula}
\label{section-Weyl-character-formula}
Highest weight representations of Kac-Moody algebras. 
Let $\lambda \in \mathfrak{h}^*$, where
$\mathfrak{g}(A)=\mathfrak{n}_-\oplus\mathfrak{h}\oplus \mathfrak{n}_+$ 
is a Kac-Moody algebra. We define a {\it Verma module} 
$$
M(\Lambda)
=U(\mathfrak{g}) \otimes_{U(\mathfrak{h}+\mathfrak{n}_+}\mathbb{C}v_\Lambda
$$
where $\mathfrak{h}+\mathfrak{n}_+$ acts on $V_\Lambda$ by:
\begin{align*}
Xv_\Lambda&=0,\qquad \forall x \in \mathfrak{n}_+,
Hv_\Lambda&=\Lambda(H)v_\Lambda,\qquad \forall H \in \mathfrak{h}
\end{align*}
So $\mathbb{C} v_\Lambda$ is a $U(\mathfrak{h}+\mathfrak{n}_+)$-module,
$$
\xymatrix{
U(\mathfrak{h}+\mathfrak{n}_+)\ar[d]\\
U(\mathfrak{g})
}
$$
By the PBW theorem, $M(\Lambda)$ has a linear $\mathbb{C}$-basis.

Let $\{F_{\alpha,i}:i = 1,\ldots, \dim \mathfrak{g}_\alpha\}$ be a basis of
$\mathfrak{g}_{-\alpha}$, $\forall \alpha \in \Delta_+$. 
Also choose a total order on $\Delta_+$. (Some sort of lexicographical order
that takes longer to write than to say.)
$$
F_{\alpha_1,i_1},F_{\alpha_2,i_2},\ldots,F_{\alpha_s,i_s},v_\Lambda
$$
$$
\alpha_1\leq \alpha_2\leq \ldots \leq \alpha_2\text{ and if }
\alpha_p=\alpha_{p+1}, i_p \leq  i_{p+1}
$$
We have $M(\Lambda)_\lambda=\{m|Hm=\lambda(H)m\}$ weight spaces.
$$
M(\Lambda)=\bigoplus_{\lambda \in \mathfrak{h}^*}M(\Lambda)_\lambda
$$
The vector $v_{\Lambda}$ is in $M(\Lambda)_\Lambda$ by definition,
\begin{align*}
F_{\alpha,i}V_\Lambda &\in M(\Lambda)_{\Lambda-\alpha}\\
H(Fv_\Lambda)&=\underbrace{[H,F]v_\Lambda}_{=-\alpha(H)Fv_\Lambda}
+\underbrace{FHv_\Lambda}_{=\Lambda(H)FV_\Lambda}
\end{align*}
So $\chi_{M(\Lambda)}=\sum_{\lambda \in \mathfrak{h}^*}\dim M(\Lambda)_\lambda
e^{\lambda}$
is computed by counting monomials $y$ with fixed $\sum_{i}\alpha_i$.

\begin{equation}
\label{equation-character-of-Verma-module}
\chi_{M(\Lambda)}=e^{\Lambda}\prod_{\alpha \in \Delta_+}
\frac{1}{(1-e^{-\alpha})^{\dim \mathfrak{g}_\alpha}}.
\end{equation}
The product on Eq. \ref{equation-character-of-Verma-module} is called 
{\it Weyl denominator}.
\begin{exercise}
\label{exercise-convince-yourself}
Convince yourself of this.
\end{exercise}

\begin{example}
\label{example-sl2-character}
$\mathfrak{g}=\mathfrak{sl}_2$, [Picture]
\begin{align*}
\chi_{M(\Lambda)}&=e^{\Lambda}+e^{\Lambda-\alpha}+e^{\Lambda-2\alpha}+\ldots\\
&=e^{\Lambda}(1+e^{-\alpha}+e^{-2\alpha}+\ldots\\
&=e^{\Lambda}\frac{1}{1-e^{-\alpha}}.
\end{align*}
\end{example}

\medskip\noindent
For certain $\Lambda$, $M(\Lambda)$ is {\it reducible} (i.e. there exists
a submodule $0 \neq N \subseteq M(\Lambda)$ (with proper contention).

\begin{lemma}
\label{lemma-}
For any submodule $N$,
$$
N=\bigoplus_{\mu \in \mathfrak{h}^*}N \cap M(\Lambda)_\mu.
$$
\end{lemma}

{\bf Corollary.} The sum of all  \underline{proper} submodules 
of  $M(\Lambda)$ is proper, in particular
there is a maximal proper submodule.

{\bf Notation.} $L(\Lambda)=M(\Lambda)\Big/\left(\substack{
\text{max. proper} \\ \text{submodule}}\right)$

\begin{example}
\label{example-}
$\mathfrak{sl}_2$. $\Lambda=3\omega$ ($\omega$: fundamental weight,
$\alpha=2\omega$.) 
$L(3\omega)=\mathbb{C}\left<e^{3\omega},e^{-\omega},e^{-3\omega}\right>$.

[Picture]
\end{example}

\begin{definition}
\label{definition-integrable-g-module}
A $\mathfrak{g}$-module is {\it integrable} if
\begin{itemize}
\item $V=\bigoplus_{\mu \in \mathfrak{h}^*}V_\mu$ (weight module).
\item For all simple roots $\alpha_i$; $e_i$ and $f_i$ are
locally nilpotent on $V$ (i.e. for all $v \in V$
there exists $N$ such that
$e_i^Nv=f_i^Nv=0$.)
\end{itemize}
\end{definition}

\begin{remark}
\label{remark-integrability}
\begin{itemize}
\item Vermas are not integrable.
\item $\dim V< \infty \implies V$ integrable.
\item $\mathfrak{g}$ itself (Kac-Moody) is integrable.
\end{itemize}
\end{remark}

\medskip\noindent
{\bf Dominant integrable weights.} 
Let $\{\alpha_1^\vee,\ldots,\alpha_\ell^\vee\}\subset\mathfrak{h}$ be the simple
coroots.

\begin{definition}
\label{definition-dominant-integral-weights}
The {\it dominant integral weights} are the weights that pair with the coroots
to give integers:
$$
P_+=\{\lambda \in \mathfrak{h}^* :
\left<\lambda,\alpha_i^\vee\right>\in\mathbb{Z}_{\geq 0,1},
i=1,\ldots,\ell\}.
$$
\end{definition}

For $L(\Lambda)$ to be integrable, it is necessary that $\Lambda \in P_+$.

Indeed, suppose $L(\Lambda)$ is integrable.
Then $f_i^Nv_\Lambda=0$ in $L(\Lambda)$,
or rather 
$$
\underbrace{e_if_i^{N+1}v_\Lambda}
_{Kf_i^N=0} \in M(\Lambda),
$$
and $K$ can only be zero if $\left<\Lambda,\alpha_i^\vee\right>
\in \mathbb{Z}_{\geq 0}$. Applying for all $i$, we find
$\Lambda \in P_+$ is necessary.

\begin{proposition}
\label{proposition-integrable-iff-dominant-integral-weight}
$L(\Lambda)$ is integrable if and only if $\Lambda \in P_+$.
\end{proposition}

\begin{proof}
For the converse, use induction and Serre relations 
(we know the result for the highest weight, and want
to prove for others).
\end{proof}

\begin{example}
\label{example-sl3-integrability}
$\mathfrak{sl}_3$. [Picture]
\end{example}

\begin{example}
\label{example-sl_2hat-integrability}
$\widehat{\mathfrak{sl}_2}$. [Picture, $P_+$ looks like diagonal lines.]
$$
\alpha_0^\vee=K-H,\qquad \alpha_1^\vee=H\in\mathfrak{sl}_2,
\qquad \left<\delta,\alpha_i^\vee\right>=0,\quad i=0,1
$$
\end{example}

\begin{remark}
\label{remark-Verma-for-affine-Kac-Moody-algebras-
the-coefficient-of-K-is-super-important}
For affine Kac-Moody algebras, almost nothing about the structure of 
$M(\Lambda)$ depends on the coefficient of $\delta$ in $\Lambda$. 
So it's common to consider
$$
M(\Lambda)=M_k(\lambda)=M(k\Lambda_0+\lambda),\qquad \lambda \in \mathfrak{h}^*
$$
where $k$, the level of $\Lambda$, is super important.
\end{remark}

Then
$$
\underbrace{\hat{P}_+}_{
\substack{\delta\text{-coef.} \\ =0}}
=\bigcup_{k \in \mathbb{Z}_{\geq 0}}
\{k\Lambda_0+\lambda|\lambda \in P_+^k\}
\{k\Lambda_0+\lambda|\lambda \in P_+^k\}.
$$
$$
P_+^k=\{\lambda \in P_+|
\left<\lambda,\theta\right>\leq k\}\subset P_+ \text{ for }\mathfrak{g}.
$$

\medskip\noindent
Consider $V= \bigoplus_{\mu \in \mathfrak{h}^*}V_\mu$ integrable,
and $V_\lambda \neq 0$.
Let $i \in \{1,\ldots,\ell\}$.
Consider $U=\bigoplus_{n \in \mathbb{Z}}V_{\lambda+n\alpha_i}\subset V$,
and the action of $e_i,f_i$ and $h_i=[e_i,f_i]$.

$\mathfrak{sl}_2\mathbb{y}U$, locally integrable. By structure of
$\mathfrak{sl}_2$-represenations, $U$ must be finite-dimensional
with ``symmetrical'' weight space multiplicities, i.e.,
\begin{align*}
\{\lambda+n\alpha_i|n\in\mathbb{Z}\}\cap \{\text{weights of }V\}
&=\{\lambda+n\alpha_i|-p\leq n\leq q\}.
\end{align*}
and
$$
\left<\lambda-p\alpha_i,h_i\right>=-\left<\lambda+q\alpha_i,h_i\right>.
$$
\medskip\noindent
Consequently, the reflection
$r_i(\lambda):=\lambda-\left<\lambda,\alpha_i^\vee\right>\alpha_i$
has the same multiplicity as $\lambda$.

[Picture]

Now consider $M(\lambda)$ and $L(\Lambda)$ for 
$\Lambda \in P_+$.
Actually, for general $\Lambda$, $M(\Lambda)$, while not
necessarilly irreducible, has an
($\Omega$-)composition series$^*$ by 
irreducibles $L(\lambda)$.
$$
M(\Lambda)=V_0 \supset V_1 \supset V_2 \supset \ldots \supset V_n=0,
$$
such that $V_i/V_{i+1}$ is $\simeq L(\lambda_i)$ (i.e. an irreducible highest
weight module) for some $\lambda_i=\Lambda-\beta_i$, $\beta_i \in Q$. For
Kac-Moody algebras we also consider the case that $(V_i/V_{i+1})=0$ for all
$\mu \in \Omega + Q_+$, (which is the $\Omega$-composition series).

[Picture.]

For an $\Omega$-composition series, as above, we have
$$
\text{ch}_{M(\Lambda)}-\sum_{\lambda \geq \Omega}
\underbrace{[M(\Lambda):L(\lambda)]}
_{\substack{\text{\# of times $L(\lambda)$} \\ \text{
appears in the compos. series}}}
\text{ch}_{L(\lambda)}
\in \left<e^{\mu}:\mu \not \geq \Omega\right>
$$
Sending  ``$\Omega \to -\infty$'', the identity
$$
\text{ch}_{M(\Lambda)}=\sum_{\lambda \leq  \Lambda}
[M(\Lambda):L(\lambda)]\text{ch}_{L(\lambda)}
$$
makes sense.

{\bf Notation.} $b_{\Lambda,\lambda}=[M(\Lambda):L(\lambda)]$.

\begin{remark}
\label{remark-total-order-on-hstar}
Recall the partial order on weights that $\lambda \leq \Lambda$ if 
$\Lambda-\lambda \in Q= \sum \mathbb{Z}_+\alpha_i$.
$b_{\Lambda,\lambda}=1$ if $\lambda = \Lambda$ and 
$b_{\Lambda,\lambda}=0$ if not ($\lambda \leq  \Lambda$).

If we choose a total order on $\mathfrak{h}^*$, compatible with $\leq$.
Then $\{b_{\Lambda,\lambda}\}$ is a lower triangular matrix with 1 
on the diagonal.
We an define $\{m_{\Lambda,\lambda}\}$ the {\it inverse matrix}.
It's again lower triangular, 1 on the diagonal, and all
$m_{\Lambda,\lambda}$ are \underline{integers} (maybe negative now). 
And we have
\begin{equation}
\label{equation-character}
\text{ch}_{L(\Lambda)}=\sum_{\lambda \leq  \Lambda}m_{\Lambda,\lambda}
\text{ch}_{M(\lambda)}.
\end{equation}
\end{remark}

\begin{example}
\label{example-sl2-characters}
$\mathfrak{sl}_2$. $M(3)\underset{L(3)}{\supset}M(-5)
\underset{L(-5)}{\supset}0$. Since $M(-5)$ is already irreducible.
[Missing…]
\end{example}

We want to discover $m_{\Lambda,\lambda}$. 
In general massively difficult.
For $\Lambda \in P_+$, $m_{\Lambda,\lambda}$ easy.
Multiply Eq. \ref{equation-character} by  $R$
\begin{align*}
R\text{ch}_{L(\Lambda)}&=\sum_{\lambda \geq  \Lambda} m_{\Lambda,\lambda}
e^{\lambda}.
e^{\rho}R\text{ch}_{L(\Lambda)}=\sum_{\lambda \leq  \Lambda}
m_{\Lambda,\lambda}e^{\lambda+\rho}.
\end{align*}
What's $\rho$? It's $\rho \in \mathfrak{h}^*$ chosen so that
$\left<\rho,\alpha_i^\vee\right>=1$, and it's called 
the {\it Weyl vector}.

\begin{remark}
\label{remark-finite-dimension}
For $\mathfrak{g}$ finite-dimensional, $\rho=\sum_{i=1}^\ell \omega_i$ 
necessarily. (And equals $\frac{1}{2}\sum_{\alpha \in \Delta_+}\alpha$.

For $\mathfrak{g}$ finite dimensional and the affine $\hat{\mathfrak{g}}$,
$\hat{\rho}=h^\vee\Lambda_0+\rho$ works.
\end{remark}

For $w \in W=\left<r_i|i=1,\ldots,\ell\right>$,
define $w(\text{ch}_V)=\sum \dim V_\mu e^{W(\mu)}$.
We saw $w(\text{ch}_V)=\text{ch}_V$ if $V$ integrable.
In particular $w(\text{ch}_{L(\Lambda)}=\text{ch}_{L(\Lambda)}$,
$\Lambda \in P_+$.

\begin{lemma}
\label{lemma-for-Weyl-formula}
$m_{\Lambda,\lambda}=0$ unless
$\lambda+\rho=w(\Lambda+\rho)$ for some $w \in W$
\end{lemma}

\medskip\noindent
{\bf Claim.} $r_i(e^{\rho} R)=-e^\rho R$. 
So $w(e^\rho R)=\det(w)e^\rho R$ for all $w \in W$.

\begin{proof}
$$
R=\prod_{\alpha \in \Delta_+}(1-e^{-\alpha})^{\text{mult}(\alpha)}.
$$
Note that
\begin{enumerate}
\item $\text{mult}(r_i(\alpha))=\text{mult}(\alpha)$ 
for all $\alpha \in \Delta$. (Since $\mathfrak{g}$ is integrable!)
\item $r_i(\Delta_+)=\{-\alpha_i\}\cup (\Delta_+\setminus\{\alpha_i\}$.
\end{enumerate}
Any $\alpha \in \Delta_+$ is of the form
$\alpha = \sum_{i=1}^\ell k_i \alpha_i$.
If $\alpha \neq \alpha_i$, some $k_{j_0}\neq 0$, $j_0 \neq  i$ and
\begin{align*}
r_i(\alpha)&=\sum_{j}k_j \alpha_j
\left<\alpha,\alpha_i^\vee\right>\alpha_i\\
&=\sum k_j' \alpha_j\\
&=e^\rho(e^{-\alpha_i}-1)
\left(\prod_{\alpha \in \Delta_+ \setminus \alpha_i}\right)\\
&=-e^\rho R.
\end{align*}
for $k_{j_0}'=k_{j_0}>0$. Can't have a mixture of signs,
so $r_i(\alpha) \in \Delta_{\perp}$.

So
\begin{align*}
r_i(e^\rho R)&=\prod_{\alpha \in \Delta_+\setminus \alpha_i}
(1-e^{-\alpha})^{\text{mult}(\alpha)}
\cdot (1-e^{t \alpha_i}\cdot e^{\rho-\alpha_i}
\end{align*}
$$
r_i=\rho-\left<\alpha_i^\vee,\rho\right>\alpha_i=\rho-\alpha_i
$$
Hence
$$
\sum_{\lambda \leq  \Lambda}m_{\Lambda,\lambda}e^{\lambda+\rho}
$$
is $W$-skew-invariant. (Lemma \ref{lemma-for-Weyl-formula}.)

Hence
$$
\sum_{\lambda \leq  \Lambda}m_{\Lambda,\lambda}e^{\lambda+\rho}
=\sum_{w \in W}\det(w)\cdot e^{w(\Lambda+\rho)}
$$
In conclusion, the {\it Weyl character formula} 
$$
\text{ch}_{L(\Lambda)}=\sum_{w \in W}\det(w)
\cdot \frac{e^{w(\Lambda+\rho)-\rho}}{R}
$$
\end{proof}

\medskip\noindent
{\bf Corollary (Weyl denominator formula).}
$$
e^{\rho}R=\sum_{w \in W}\det(w)e^{w(\rho)}.
$$
Next time: Affine case, $\theta$-functions. Modular forms (Poisson summation.)


\bibliography{my}
\bibliographystyle{amsalpha}

\end{document}

