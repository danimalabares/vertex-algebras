\input{preamble}

\begin{document}

\title{Vertex Algebras}
\maketitle

\phantomsection
\label{section-phantom}
\hfill
\href{http://github.com/danimalabares/vertex-algebras}
{github.com/danimalabares/vertex-algebras}

\tableofcontents

\section{Kac-Moody algebras}
\label{section-Kac-Moody-algebras}

Kac-Moody algebras are Lie algebras, whose definition is motivated by the
structure of finite-dimensional simple Lie algebras over $\mathbb{C}$.

Let $\mathfrak{g}$ be a finite-dimensional semisimple Lie algebra over
$\mathbb{C}$. Then $\mathfrak{g}$ has a {\it Cartan subalgebra} 
$\mathfrak{h}\subset \mathfrak{g}$ (abelian + …). 
Fixing $\mathfrak{h}\subset\mathfrak{g}$ gives a 
{\it root space decomposition} 
$$
\mathfrak{g}=\mathfrak{h}\oplus \bigoplus_{\alpha \in \Delta}\mathfrak{g}_\alpha
$$
where $\Delta \subset \mathfrak{h}^*$ linear dual, and, by definition
$$
\mathfrak{g}_\alpha=\{X \in \mathfrak{g}|
[H,X]=\alpha(H)X\; \forall H \in \mathfrak{h}\}
$$
Turns out the $\mathfrak{g}_\alpha$ are all 1-dimensional, 
though this property is lost when we go to Kad-Moody algebras.
$$
[\mathfrak{g}_\alpha,\mathfrak{g}_\beta] \subset \mathfrak{g}_{\alpha+\beta}
$$
The Killing form  $\kappa:\mathfrak{g} \times \mathfrak{g} \to \mathbb{C}$,
 $\kappa(x,y)=\text{Tr}_\mathfrak{g}\text{ad}(x)\text{ad}(y)$ is nondegenerate. 
``This is kind of the definition of semisimple.'' 
(Think of $\mathfrak{h}$ as $\mathfrak{g}_0$, btw.)

$\kappa |_{\mathfrak{g}_\alpha \times \mathfrak{g}_\beta}\neq 0$ only when 
$\beta=-\alpha$. $\kappa |_{\mathfrak{h}\times \mathfrak{h}}$ is non-degenerate. 
 This gives a linear isomorphism 
$\mathfrak{h} \xrightarrow{\nu} \mathfrak{h}$ via 
$\nu(H)(H') = \kappa(H,H')$.

\medskip\noindent
So, $\mathfrak{h}^*$ comes with a non-degenerate bilinear form.

The {\it reflection} $r_\alpha:\mathfrak{h}\to \mathfrak{h}^*$ in 
$\alpha \in \mathfrak{h}^*$ (usually a root) is 
$r_\alpha(\lambda)=\lambda- 2\frac{(\lambda,\alpha)}{(\alpha,\alpha)}
\cdot \alpha.$

``Classify root systems […] classify semisimple Lie algebras'' 
It is a fact that $r_\alpha (\Delta)=\Delta$ for all $\alpha \in \Delta$, 
which motivates the definition of {\it root system} and permits classification.

\begin{example}
\label{example-root-system-sl2}
$\mathfrak{g}=\mathfrak{sl}_2$, $\mathfrak{h}=$ diagonal matrices
$$
\begin{pmatrix}
1&&\\ 
&-1&\\
& & 0
\end{pmatrix}
\qquad 
\begin{pmatrix}
0&&\\ 
&1&\\
& & -1
\end{pmatrix}
$$
is a basis of $\mathfrak{h}$. There are 6 roots vectors
$$
E_{12}=
\begin{pmatrix}
0 & 1 & 0\\
0 & 0 & 0\\
0 & 0 & 0
\end{pmatrix}
$$
$E_{23},E_{13}$, etc.
\end{example}

\begin{exercise}
\label{exercise-basis-of-h}
$[H_1,E_{12}]=2E_{12}$, $[H_2,E_{12}]=-E_{12}$, $\alpha_{12}=(2,-1)$.
\end{exercise}

[Drawing of roots]

Notions of {\it positive roots} and {\it simple roots} 
(set of $\text{rank}\mathfrak{g}g$ simple roots has $\ell$ elements,
 where $\ell=\dim(\mathfrak{h}^*)$. 
This will also fail for Kac-Moody algebras more generally). 
Next write the Cartan matrix
$$
A=(a_{ij}),\qquad  a_{ij}=2\frac{)\alpha_i,\alpha_j)}{(\alpha_i,\alpha_j)}
$$
for $1\leq i,j \leq \ell$.

\begin{example}
\label{example-Cartan-matrix-sl3}
$\mathfrak{sl}_3$. [Picture, hexagonal pattern]. 
$(\alpha_1,\alpha_1)=(\alpha_2,\alpha_2)=2,(\alpha_1,\alpha_2)=-1$, so
$$
A=\begin{pmatrix}
2&-1\\ 
-1&2
\end{pmatrix}
$$
\end{example}

\begin{example}
\label{example-Cartan-matrix-sl5}
$\mathfrak{sl}_5$. [Picture, square pattern]. $|\alpha_2|=1$, $|\alpha_1|=2$,
$(\alpha_1,\alpha_2)=-2$, so
$$
A=\begin{pmatrix}
2&-1\\ 
-2&2
\end{pmatrix}
$$
\end{example}

\begin{remark}
\label{remark-}
Since $\mathfrak{g}_\alpha$ is 1-dimensional, set
$\mathfrak{g}_\alpha=\mathbb{C}E_\alpha$ and 
$E_i=E_{\alpha_i}$, $i=1,2,\ldots,\ell$ (simple root vectors). 
It turns out that 
$$
\text{ad}(E_i)^{1-a_{ij}}E_j=0.
$$
This is called a {\it Serre relation}.
\end{remark}

\section{Some infinite dimensional Lie algebras}
\label{section-some-infinite-dimensional-Lie-algebras}
Let $\mathfrak{g}$ be a finite-dimensional semisimple Lie algebra, and define
the {\it loop algebra} 
\begin{align*}
L\mathfrak{g}&=\mathfrak{g}[t,t^{-1}],\text{ (with basis 
$at^m|\substack{\text{$a \in $ a basis of $\mathfrak{g}$} \\ m \in \mathbb{Z}}$
)}\\
&=\mathfrak{g} \otimes _{\mathbb{C}}\mathbb{C}[t,t^{-1}]
\end{align*}
with the Lie bracket
$$
[at^m,bt^n]=[a,b]t^{m+n}.
$$
``This construction is absurdely general --- we don't need $\mathfrak{g}$ to be
semisimple […]''

\medskip\noindent
Take $\mathfrak{g}=\mathfrak{sl}_2$. Recall that
$$
E=\begin{pmatrix}
0&1\\ 
0&0
\end{pmatrix},\quad 
H=\begin{pmatrix}
1&0\\ 
0&-1
\end{pmatrix},\quad 
F=\begin{pmatrix}
0&0\\ 
1&0
\end{pmatrix}
$$
[Picture with $F,H,E,Ft,Ht,Et,Et^2$…]
$E$ was a root vector, corresponding to the unique root in $\mathfrak{sl}_2$,
call it $\alpha_1$. We seem to have a second simple root $\alpha_0$,
 corresponding to $Ft$.

This looks like it wants to have a Cartan matrix
 $$
A=\begin{pmatrix}
2&-2\\ 
-2&2
\end{pmatrix}
$$
We will indeed recover (a variant of) $L\mathfrak{g}$ as a Lie algebra 
``built from'' $A=\begin{pmatrix}
2&-2\\ 
-2&2
\end{pmatrix}$,
 a Kac-Moody algebra. 
But note first, $\mathfrak{h}=\mathbb{C}H$ is too small.
``Problem with $\alpha_0$ and $\alpha_1$ being linearly independent  …''

We can consider $L\mathfrak{g}\oplus\mathbb{C}d$, and set 
$[d,at^m]=mat^n$ 

\begin{exercise}
\label{exercise-this-defines-a-Lie-algebra}
Check this defines a Lie algebra.
\end{exercise}

The Kac-Moody algebra turns out to be, not quite this, but slightly larger
still.

\medskip\noindent
\begin{definition}
\label{definition-}
Given $\mathfrak{g}$ simple, with 
$(\cdot,\cdot):\mathfrak{g} \times \mathfrak{g} \to \mathbb{C}$ 
invariant bilinear form, there is a Lie algebra 
$$
\hat{\mathfrak{g}}=L\mathfrak{g} \oplus \mathbb{C}K,
$$
with $[K,\hat{\mathfrak{g}}]=0$, and 
$[at^m,bt^n]=[a,b]t^{m+n}+m(a,b)\delta_{m_1}-nK$.
$[K,\hat{g}]=0$, $K$: central.

``For the construction to work it doesn't actually have to be nondegenerate.''

This is called an {\it affine Lie algebra}. We also have
$\tilde{\mathfrak{g}}=L\mathfrak{g} \oplus \mathbb{C}K \oplus \mathbb{C}d$, 
{\it extended affine Lie algebra}, $[d,at^m]=m a t^m$ as before, and 
$[K,d]=0$.
\end{definition}

The extended affine Lie algebra is an example of a Kac-Moody algebra.

\section{Kac-Moody algebras}
\label{section-Kac-Moody-algebras-again}

Recall the notion of the free Lie algebra on a 
vector space $V$ of generators 
(or a set $X$, think of $V$ as a vector space with basis $X$):

\begin{definition}
\label{definition-free-Lie-algebra}
The {\it free Lie algebra} on $V$ is characterized by the universal property
$$
\xymatrix{
V\ar[rr]^f\ar[dr]^i&&\mathfrak{g}\\
& F(V)\ar[ur]_{\exists !\tilde{f}}
}
$$
That is, for all linear map $f:V \to \mathfrak{g}$ with $\mathfrak{g}$ Lie
algebra, there exists a unique $\tilde{f}$ homomorphism of Lie algebras 
$F(V)\to \mathfrak{g}$ such that $\tilde{f} \circ i=f$.
$$
\Hom_{\text{Lie}}(F(V),\mathfrak{g})=\Hom_{\text{Vec}}(V,\mathfrak{g})
$$
naturally
\end{definition}

That is, $F$ and the 
forgetful functor $G:\underline{\text{Lie}}\to \underline{\text{Vec}}$
are adjoint:
$$
\Hom_{\underline{\text{Lie}}}(F(V),\mathfrak{g})
\xrightarrow{\simeq }
\Hom_{\underline{\text{Vec}}}(V,G(\mathfrak{g}))
$$

\medskip\noindent
{\bf A realisation of $F(V)$.} Let 
$$
T(V)=\mathbb{C} \oplus V \oplus V^{\otimes 2}\oplus V^{\otimes 3}\oplus\ldots
$$
be the tensor algebra of $V$. 

Then inside $T(V)$ consider $F(V)$ the span of iterated commutators of elements
of  $V$.

\begin{proposition}
\label{proposition-this-realises-the-free-Lie-algebra}
This realises the free Lie algebra.
\end{proposition}

\begin{proof}
In online notes.
\end{proof}

\medskip\noindent
In the finite dimensional simple case, we had
$$
a_{ij}=\frac{2(\alpha_i,\alpha_j)}{(\alpha_i,\alpha_i},
$$
which we think also as $\alpha_i,\alpha_j \in \mathfrak{h}^*$, 
and $\alpha_i^\vee=\frac{2}{(\alpha_i,\alpha_i)}\nu^{-1}(\alpha_i) 
\in \mathfrak{h}.$

Clearly, $\alpha_{ii}=2$ for all $i$. 
$a_{ij}$ misht not equal $a_{ji}$, but certainly $a_{ij}=0 \iff a_{ji}=0$. 
And $\forall  i \neq j$, $a_{ij} \leq 0$.

\medskip\noindent
{\bf One further property.} Set
$$
\varepsilon_i=\frac{2}{(\alpha_i,\alpha_i)},\quad \text{and}\quad 
D=\substack{\text{diagonal matrix} \\ \text{with entries }\varepsilon_i}
$$
Then $A=DB$, where $B=((\alpha_i,\alpha_j))$ is symmetric. 
If a matrix $A$ is equal to $(\text{diag})(\text{symm})$, we call it
 {\it symmetrizable}.

\begin{definition}
\label{definition-generalized-Cartan-matrix}
A {\it generalized Cartan matrix} is an integer matrix $A=(a_{ij})$ 
which is
\begin{itemize}
\item symmetrizable,
\item $a_{ii}=2$ for all $i$,
\item $a_{ij}=0 \iff a_{ji}=0$,
\item $a_{ij}\leq 0$ for $i \neq j$.
\end{itemize}
\end{definition}

\begin{definition}
\label{definition-realisation}
A {\it realisation} of a generalized Cartan matrix is a complex vector space
 $\mathfrak{h}$, and two sets
\begin{align*}
\Pi^\vee&=\{\alpha_1^\vee, \alpha_2^\vee,\ldots,\alpha_n^\vee\},\quad
\text{and},\\
\Pi&=\{\alpha_1,\alpha_2,\ldots,\alpha_n\}
\end{align*}
such that $\left<\alpha_i^\vee,\alpha_j\right>=a_{ij}$, $1\leq i,j\leq n$.
\end{definition}

\begin{exercise}
\label{exercise-realisation}
$\dim(\mathfrak{h})\geq 2n-\text{rank}(A)$.
\end{exercise}

\begin{proof}
For $A=\begin{pmatrix}
2&-2\\ 
-2&2
\end{pmatrix}$, a realisation is given by
$$
\Pi^\vee=\{H_1,H_0\},\qquad \Pi=\{\alpha_0,\alpha_1\}
$$
\begin{align*}
\mathfrak{h}&=\mathbb{C}H,\mathbb{C}d,\mathbb{C}K,\\
\mathfrak{h}^*&=\mathbb{C}\alpha_1+\mathbb{C}\delta+\mathbb{C}\Lambda_0
\end{align*}
(Canonical dual, $\left<\alpha_1,H\right>=2$, 
$\left<\delta,d\right>=1=\left<\Lambda_0,K\right>$, every other pairing $0$.)

Then
$$
\begin{cases}
\alpha_1=\alpha_1\\
\alpha_0=\delta-\alpha_1
\end{cases}\qquad 
\begin{cases}
\alpha_1^\vee=H \\
\alpha_0^\vee = K-H
\end{cases}
$$
So we obtain
\begin{align*}
\left<\alpha_0^\vee,\alpha_1\right>&=\left<K-H,\alpha_1\right>=2\\
\left<\alpha_1^\vee,\alpha_0\right>&=\left<H,\delta-\alpha_1\right>=-2\\
\left<\alpha_0^\vee,\alpha_0\right>&=\left<K-H,\delta-\alpha_1\right>=+2
\end{align*}

\end{proof}

\medskip\noindent
Finally let's define Kac-Moody algebras.

Let $A$ be a generalized Cartan matrix. Let 
$$
\tilde{\mathfrak{n}}_+=F(e_1,\ldots,e_n),
$$ 
free Lie algebra on $n$ generators, and similarly
$$
\tilde{\mathfrak{n}}_-=F(f_1,\ldots,f_n)
$$
Let $\mathfrak{h}$ be a realisation of $A$. 
Set $\tilde{\mathfrak{g}}(A)
=\tilde{\mathfrak{n}}_- \oplus \mathfrak{h} \oplus \tilde{\mathfrak{n}}_+$.

Make $\tilde{\mathfrak{g}}(A)$ a Lie algebra by defining
\begin{itemize}
\item $[\mathfrak{h},\mathfrak{h}]=0$,
\item $\forall  H \in \mathfrak{h}$, $[H,e_i]=\left<\alpha_i,H\right>e_i
=\alpha_i(H)e_i$. And similarly, $[H,f_i]=-\alpha_i(H)f_i$.
\item $[e_i,f_j]=\delta_{ij}\alpha_i^\vee$.
\end{itemize}
Then $\tilde{\mathfrak{g}}(A)$ is a Lie algebra (though not yet the Kac-Moody
algebra). See Kac, \cite[Thm 1.2]{IDLA}.

\begin{remark}
\label{remark-lattice}
In $\mathfrak{h}$ we have a lattice
\begin{align*}
Q^\vee&=\mathbb{Z}\alpha_1^\vee+\ldots+\mathbb{Z}\alpha_n^\vee,\quad
\text{and}\\
Q&=\mathbb{Z}\alpha_1+\ldots+\mathbb{Z}\alpha_n\text{ in }\mathfrak{h}^*
\end{align*}
(root and coroot lattices). $\tilde{\mathfrak{g}}(A)$ is naturally $Q$-graded,
with $\tilde{\mathfrak{g}}(A)_\beta=\text{span}\{
\text{commutators of $e_i$ with $\sum \alpha_i=\beta$}\}$.
$\tilde{g}(A)=\mathfrak{h}$.

\begin{theorem}[Gabber-Kac]
\label{theorem-Gabber-Kac}
Denote by $I \subset \tilde{\mathfrak{g}}(A)$ the maximal $Q$-graded ideal, such
that $I \cap\mathfrak{h}=\{0\}$. 
Then $I$ is generated by the Serre relations 
$$
\text{ad}(e_i)^{1-a_{ij}}e_j\qquad \text{and}\qquad 
\text{ad}(f_i)^{1-a_{ij}}f_j,\; i\neq j.
$$
\end{theorem}

\begin{proof}
\cite[Theorem 9.11]{IDLA}.
\end{proof}

\begin{definition}
\label{definition-Kac-Moody-algebra}
The {\it Kac-Moody algebra} $\mathfrak{g}(A)$ is $\tilde{\mathfrak{g}}(A)/I$.
\end{definition}

\end{remark}


\bibliography{my}
\bibliographystyle{amsalpha}

\end{document}

